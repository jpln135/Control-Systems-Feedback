\begin{enumerate}[label=\thesubsection.\arabic*.,ref=\thesubsection.\theenumi]
\numberwithin{equation}{enumi}
\item
Consider a Feedback Current Amplifier formed by cascading an Inverting Opamp $\mu$ with a MOSFET (NMOS).
The output current is the Drain Current of the NMOS.
Assume that Opamp has an input resistance $R_{id}$, an Open Circuit Voltage Gain $\mu$, and an output resistance $r_{o1}$

\begin{figure}[!ht]
	\begin{center}
		
		\resizebox{\columnwidth}{!}{\ctikzset{tripoles/mos style/arrows}
\usetikzlibrary{arrows.meta,decorations.markings}
\begin{circuitikz}
\ctikzset{bipoles/length=1cm}
\draw
(0, 0) node[op amp] (opamp) {$\mu$}
(1.5,0) node[nmos,](Q){}
(opamp.-) -- (-2,0.35) -- (-2,-2) to[R=$R_2$,i=$I_{f}$] (1.5,-2) --(Q.S){}
(1.5,-0.2) to[short, i=$I_{0}$] (1.5, -2)
(1.5,-2) to[R=$R_1$] (1.5,-4) node[ground]{}
(opamp.+) -- (-1,-0.35) -- (-1,-0.4) node[ground]{}
(Q.center) node[right]{{$Q$}}
(Q.D) -- (1.5,1)-- (2,1) node[ground]{}
(-2,-2) --(-3,-2) to[R=$R_s$] (-3,-4) node[ground]{}
(-4,-4) node[ground]{}
(-4,-4) to[I=$I_s$] (-4,-2) --(-3,-2)
;\end{circuitikz}}
	\end{center}
\caption{}
\label{fig:Complete_Circuit}
\end{figure}

\item
Represent the Control System using a block diagram

\solution
\begin{figure}[!ht]
	\begin{center}
			\resizebox{\columnwidth}{!}{\tikzstyle{input} = [coordinate]
\tikzstyle{output} = [coordinate]
\tikzstyle{block} = [draw, rectangle]
\tikzstyle{sum} = [draw, circle]

\begin{tikzpicture}[auto, node distance=2cm,>=latex']
    \node [input, name=input] {$I_s$};
    \node [sum, right of=input] (sum) {};
    \node [block, right of=sum] (controller) {$G$};
    \node [output, right of=controller] (output) {};
    \node [block, below of=controller] (feedback) {$H$};
    \draw [draw,->] (input) -- node {$I_{s}$} (sum);
    \draw [->] (sum) -- node {} (controller);
    \draw [->] (controller) -- node [name=y] {$I_o$}(output);
    \draw [->] (y) |- (feedback);
    \draw [->] (feedback) -| node[pos=0.95]{$-$}  node [near end] {$I_f$} (sum);
    \draw [->] (feedback) -| node[pos=1.15]{$+$}  node [near end] {} (sum);
\end{tikzpicture}}
	\end{center}
\caption{}
\label{fig:Block_Diagram}
\end{figure}

\item
If loop gain is large, find approximate expression for closed loop gain $T$

\solution
Given,
\begin{align}
    GH \gg 1
\end{align}
\begin{align}
    T = \frac{G}{1+GH}\simeq \frac{1}{H}
\end{align}

\begin{align}
    H = \frac{I_{f}}{I_{o}}=-\frac{R_{1}}{R_{1}+R_{2}}
\end{align}
\begin{align}
    T \simeq \frac{1}{H}=-\left(1+\frac{R_{2}}{R_{1}}\right)
\end{align}

\item
Find the G Circuit i.e the Gain Circuit and approximate expressions for G, $R_{i}$, $R_{o}$

\solution
By replacing the Opamp with its equivalent model we can get the G circuit

\begin{figure}[!ht]
	\begin{center}
		
		\resizebox{\columnwidth}{!}{\begin{circuitikz}
\ctikzset{tripoles/mos style/arrows}
\draw
(0,0) node[nmos,](Q){}
(Q.center) node[right]{{$Q$}}
(Q.D) -- (0.75,0.75) to[R=$r_{02}$] (0.75,-0.75)-- (Q.S)
(Q.D) -- (0,1.5) -- (0.5,1.5) node[ground]{}
(Q.G) -- (-1,0) to[R=$r_{01}$, i = 0] (-4,0) 
(-4,-3.5) node[ground]{}
(-4,-3.5) to[V = $\mu V_i$] (-4,0){}
(Q.S) -- (0,-1.5) to[R=$R_1$] (0,-3.5) node[ground]{}
(0,-1.5) --(-1,-1.5) to[R=$R_2$] (-1,-3.5) node[ground]{}
;\end{circuitikz}}
	\end{center}
\caption{}
\label{fig:Complete_Circuit}
\end{figure}


\begin{align}
    R_{i}=R_{s}\|R_{i d}\|(R_{1}+R_{2})
\end{align}
\begin{align}
    V_{i}=I_{i} R_{i}
\end{align}
\begin{align}
    I_{o}=-\mu V_{i} \frac{1}{1 / g_{m}+(R_{1}\|R_{2}\| r_{o 2})} \frac{r_{o 2}}{r_{o 2}+(R_{1} \| R_{2})}
\end{align}
\begin{align}
    G = \frac{I_{o}}{I_{i}}=-\mu \frac{R_{i}}{1 / g_{m}+(R_{1}\|R_{2}\| r_{o 2})} \frac{r_{o 2}}{r_{o 2}+(R_{1} \| R_{2})}
\end{align}
We use the approximation
\begin{align}
    1 / g_{m} \ll (R_{1}\|R_{2}\| r_{o 2})
\end{align}
This is because the $\frac{1}{g_{m}}$ is in order of few \ohm s but, $R_{1}$, $R_{2}$ and $r_{o2}$ are in order of k\ohm s 

\begin{align}
    G =-\mu \frac{R_{i}}{R_{1} \| R_{2}}
\end{align}
\begin{align}
    R_{o}=r_{o 2}+(R_{1} \| R_{2})+(g_{m} r_{o 2})(R_{1} \| R_{2})
\end{align}
\begin{align}
    \implies R_{o} \simeq g_{m} r_{o 2}\left(R_{1} \| R_{2}\right)
\end{align}

\item
Give expressions for GH, $T$, $R_{if}$, $R_{in}$, $R_{of}$, $R_{out}$

\solution
\begin{align}
    GH=\mu \frac{R_{i}}{\frac{1}{g_{m}}+(R_{1}\|R_{2}\| r_{o 2})} \frac{r_{o 2}}{r_{o 2}+(R_{1} \| R_{2})} \frac{R_{1}}{R_{1}+R_{2}}
\end{align}

Once again, using the approximation,
\begin{align}
    \implies GH \simeq \mu \frac{R_{i}}{R_{1} \| R_{2}} \frac{R_{1}}{R_{1}+R_{2}}=\mu \frac{R_{i}}{R_{2}}
\end{align}

For Input Resistance,
\begin{align}
    R_{if}=R_{i} /(1+GH)
\end{align}
\begin{align}
    \implies \frac{1}{R_{i f}}=\frac{1}{R_{i}}+\frac{\mu}{R_{2}}
\end{align}
\begin{align}
    \implies R_{i f}=R_{i} \| \frac{R_{2}}{\mu}
\end{align}

Substituting the value of $R_{i}$,
\begin{align}
    R_{if}=R_{s}\|R_{id}\|(R_{1}+R_{2}) \| \frac{R_{2}}{\mu}
\end{align}
\begin{align}
    R_{if}=R_{s} \| R__{in}
\end{align}
\begin{align}
    \implies R_{in}=R_{i d}\|(R_{1}+R_{2})\| \frac{R_{2}}{\mu}
\end{align}
\begin{align}
    R_{in} \simeq \frac{R_{2}}{\mu}
\end{align}

For Output Resistance,
\begin{align}
    R_{of}=R_{o}(1+GH) \simeq GH R_{o}
\end{align}
\begin{align}
    R_{of} \simeq \mu (\frac{R_{i}}{R_{2}})(g_{m} r_{o 2})(R_{1} \| R_{2})
\end{align}
\begin{align}
    R_{out} = R_{of}=\mu \frac{R_{i}}{R_{1}+R_{2}}(g_{m} r_{o 2}) R_{1}
\end{align}

\item
Find numerical values for G, H, GH, T, $R_{i}$, $R_{if}$, $R_{in}$, $R_{o}$,  $R_{of}$, $R_{out}$ given the following values
$\mu$ = 1000, $R_{s}$ = \inf, $R_{id}$ = \inf, $r_{o1}$ = 1 k\ohm,  $R_{1}$ = 10 k\ohm, $R_{2}$ = 90k\ohm

\solution
Using the given numerical values on the previously obtained equations, we obtain:
\begin{align}
    R_{i}=\infty\|\infty\|(10+90)=100 k\ohm
\end{align}

We can see that our approximation is valid
\begin{align}
    1 / g_{m}=0.2 k\ohm \ll (10\|90\| 20)k\ohm = 6.2k\ohm 
\end{align}
\begin{align}
    G =-1000 \frac{100}{10 \| 90}=-11.11 \times 10^{3}
\end{align}
\begin{align}
    H=-\frac{R_{1}}{R_{1}+R_{2}}=-\frac{10}{10+90}=-0.1
\end{align}
\begin{align}
    GH=1111
\end{align}
\begin{align}
    T \simeq \frac{1}{H} = -\frac{1}{0.1} = -10
\end{align}
\begin{align}
    R_{in}=\frac{R_{2}}{\mu}=\frac{90k\ohm}{1000}=90\ohm
\end{align}
\begin{align}
    R_{o} &=g_{m} r_{o 2}(R_{1} \| R_{2}) =5 \times 20(10 \| 90)=900k\ohm
\end{align}
\begin{align}
    R_{out}=(1+GH) R_{o}=1112 \times 900 \simeq 1000M\ohm
\end{align}

\end{enumerate}