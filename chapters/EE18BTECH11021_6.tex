\begin{enumerate}[label=\thesubsection.\arabic*.,ref=\thesubsection.\theenumi]
\numberwithin{equation}{enumi}
\item
Consider a Feedback Current Amplifier formed by cascading an Inverting Opamp $\mu$ with a MOSFET (NMOS).
The output current is the Drain Current of the NMOS.
Assume that Opamp has an input resistance $R_{id}$, an Open Circuit Voltage Gain $\mu$, and an output resistance $r_{o1}$

\begin{figure}[!ht]
	\begin{center}
		\resizebox{\columnwidth}{!}{\ctikzset{tripoles/mos style/arrows}
\usetikzlibrary{arrows.meta,decorations.markings}
\begin{circuitikz}
\ctikzset{bipoles/length=1cm}
\draw
(0, 0) node[op amp] (opamp) {$\mu$}
(1.5,0) node[nmos,](Q){}
(opamp.-) -- (-2,0.35) -- (-2,-2) to[R=$R_2$,i=$I_{f}$]
(1.5,-2) --(Q.S){}
(-2,-2) to[short, i=$I_{i}$] (-2,0.35)
(1.5,-0.2) to[short, i=$I_{0}$] (1.5, -2)
(1.5,-2) to[R=$R_1$] (1.5,-4) node[ground]{}
(opamp.+) -- (-1,-0.35) -- (-1,-0.4) node[ground]{}
(Q.center) node[right]{{$Q$}}
(Q.D) -- (1.5,1)-- (2,1) node[ground]{}
(-2,-2) --(-3,-2) to[R=$R_s$] (-3,-4) node[ground]{}
(-4,-4) node[ground]{}
(-4,-4) to[I=$I_s$] (-4,-2) --(-3,-2)
;\end{circuitikz}}
	\end{center}
\caption{Complete Circuit}
\label{fig:Complete_Circuit}
\end{figure}

\item
Represent the given circuit using a Small Signal Equivalent Model.

\solution
\begin{figure}[!ht]
	\begin{center}
		\resizebox{\columnwidth}{!}{\begin{circuitikz}
\ctikzset{bipoles/length=1cm}
\draw
(-0.5,0) to[short,i=$I_i$,*-*] (0.5,0){}
(0.5,0) -- (0.5,-1) to[R=$R_s$] (0.5,-2) --(0.5,-3) node[ground]{}
(0.5,0) -- (1.5,0) -- (1.5,-1) to[R=$R_{id}$] (1.5,-2) --(1.5,-3) node[ground]{}
(1.5,0) -- (2.5,0) -- (2.5,-0.5) to[R=$R_1$] (2.5,-2) to[R=$R_2$] (2.5,-2.5) --(2.5,-3) node[ground]{}
(2.5,0) -- (3.5,0) node at(3.5,-0.2){$+$}
node at(3.5,-3){$-$}
node at(3.5,-1.25){$V_i$}
(5,0) to[R=$r_{01}$,*-*] (7,0){}
(5,-3) node[ground]{}
(5,-3)--(5,-2) to[V=$\mu V_i$]  (5,-1) -- (5,0){}
node at(7,-0.2) {$+$}
node at(7,-3) {$-$}
node at(7,-2) {$V_{gs}$}
(8.5,0) to[I=$g_{m}V{gs}$] (8.5,-2){}
(8.5,0)--((8.5,1) --(9,1) node[ground]{}
(8.5,0) -- (10,0) to[R=$r_{02}$] (10,-2) -- (8.5,-2){}
(8.5,-2) to[short, i = $I_{o}$] (8.5,-3) -- (8.5,-3.5) to[R=$R_1$] (8.5,-4.5) -- (8.5,-5) node[ground]{}
(8.5,-3) -- (7.5,-3) --(7.5,-3.5) to[R=$R_2$] (7.5,-4.5) -- (7.5,-5) node[ground]{}
;\end{circuitikz}
}
	\end{center}
\caption{Small Signal Model}
\label{fig:Small_Signal_Model}
\end{figure}

\item
Represent the Control System using a block diagram

\solution
\begin{figure}[!ht]
	\begin{center}
			\resizebox{\columnwidth}{!}{\tikzstyle{input} = [coordinate]
\tikzstyle{output} = [coordinate]
\tikzstyle{block} = [draw, rectangle]
\tikzstyle{sum} = [draw, circle]

\begin{tikzpicture}[auto, node distance=2cm,>=latex']
    \node [input, name=input] {$I_s$};
    \node [sum, right of=input] (sum) {};
    \node [block, right of=sum] (controller) {$G$};
    \node [output, right of=controller] (output) {};
    \node [block, below of=controller] (feedback) {$H$};
    \draw [draw,->] (input) -- node {$I_{s}$} (sum);
    \draw [->] (sum) -- node {} (controller);
    \draw [->] (controller) -- node [name=y] {$I_o$}(output);
    \draw [->] (y) |- (feedback);
    \draw [->] (feedback) -| node[pos=0.95]{$-$}  node [near end] {$I_f$} (sum);
    \draw [->] (feedback) -| node[pos=1.15]{$+$}  node [near end] {} (sum);
\end{tikzpicture}}
	\end{center}
\caption{Block Diagram}
\label{fig:Block_Diagram}
\end{figure}

\item
Find the G Circuit i.e the Gain Circuit and approximate expressions for G, $R_{i}$, $R_{o}$

\solution
By replacing the Opamp with its equivalent model we can get the G circuit

\begin{figure}[!ht]
	\begin{center}
		
		\resizebox{\columnwidth}{!}{\begin{circuitikz}
\ctikzset{tripoles/mos style/arrows}
\draw
(0,0) node[nmos,](Q){}
(Q.center) node[right]{{$Q$}}
(Q.D) -- (0.75,0.75) to[R=$r_{02}$] (0.75,-0.75)-- (Q.S)
(Q.D) -- (0,1.5) -- (0.5,1.5) node[ground]{}
(Q.G) -- (-1,0) to[R=$r_{01}$, i = 0] (-4,0) 
(-4,-3.5) node[ground]{}
(-4,-3.5) to[V = $\mu V_i$] (-4,0){}
(Q.S) -- (0,-1.5) to[R=$R_1$] (0,-3.5) node[ground]{}
(0,-1.5) --(-1,-1.5) to[R=$R_2$] (-1,-3.5) node[ground]{}
;\end{circuitikz}}
	\end{center}
\caption{Gain Circuit}
\label{fig:Complete_Circuit}
\end{figure}


\begin{align}
    R_{i}=R_{s}\|R_{i d}\|(R_{1}+R_{2})
\end{align}
\begin{align}
    V_{i}=I_{i} R_{i}
\end{align}
\begin{align}
    I_{o}=-\mu V_{i} \frac{1}{1 / g_{m}+(R_{1}\|R_{2}\| r_{o 2})} \frac{r_{o 2}}{r_{o 2}+(R_{1} \| R_{2})}
\end{align}
\begin{align}
    G = \frac{I_{o}}{I_{i}}=-\mu \frac{R_{i}}{1 / g_{m}+(R_{1}\|R_{2}\| r_{o 2})} \frac{r_{o 2}}{r_{o 2}+(R_{1} \| R_{2})}
\end{align}
We use the approximation
\begin{align}
    1 / g_{m} \ll (R_{1}\|R_{2}\| r_{o 2})
\end{align}
This is because the $\frac{1}{g_{m}}$ is in order of few \ohm s but, $R_{1}$, $R_{2}$ and $r_{o2}$ are in order of k\ohm s 

\begin{align}
    G =-\mu \frac{R_{i}}{R_{1} \| R_{2}}
\end{align}
\begin{align}
    R_{o}=r_{o 2}+(R_{1} \| R_{2})+(g_{m} r_{o 2})(R_{1} \| R_{2})
\end{align}
\begin{align}
    \implies R_{o} \simeq g_{m} r_{o 2}\left(R_{1} \| R_{2}\right)
\end{align}

\item
Find expression for Loop Gain H

\solution
\begin{align}
    H = \frac{I_{f}}{I_{o}}=-\frac{R_{1}}{R_{1}+R_{2}}
\end{align}

\item
If loop gain is large, find approximate expression for closed loop gain $T$

\solution
Given,
\begin{align}
    GH \gg 1
\end{align}
\begin{align}
    T = \frac{G}{1+GH}\simeq \frac{1}{H}
\end{align}


\begin{align}
    T \simeq \frac{1}{H}=-\left(1+\frac{R_{2}}{R_{1}}\right)
\end{align}

\item
Give expressions for GH, $T$, $R_{if}$, $R_{in}$, $R_{of}$, $R_{out}$

\solution
\begin{align}
    GH=\mu \frac{R_{i}}{\frac{1}{g_{m}}+(R_{1}\|R_{2}\| r_{o 2})} \frac{r_{o 2}}{r_{o 2}+(R_{1} \| R_{2})} \frac{R_{1}}{R_{1}+R_{2}}
\end{align}

Once again, using the approximation,
\begin{align}
    \implies GH \simeq \mu \frac{R_{i}}{R_{1} \| R_{2}} \frac{R_{1}}{R_{1}+R_{2}}=\mu \frac{R_{i}}{R_{2}}
\end{align}

For Input Resistance,
\begin{align}
    R_{if}=R_{i} /(1+GH)
\end{align}
\begin{align}
    \implies \frac{1}{R_{i f}}=\frac{1}{R_{i}}+\frac{\mu}{R_{2}}
\end{align}
\begin{align}
    \implies R_{i f}=R_{i} \| \frac{R_{2}}{\mu}
\end{align}

Substituting the value of $R_{i}$,
\begin{align}
    R_{if}=R_{s}\|R_{id}\|(R_{1}+R_{2}) \| \frac{R_{2}}{\mu}
\end{align}
\begin{align}
    R_{if}=R_{s} \| R_{in}
\end{align}
\begin{align}
    \implies R_{in}=R_{i d}\|(R_{1}+R_{2})\| \frac{R_{2}}{\mu}
\end{align}
\begin{align}
    R_{in} \simeq \frac{R_{2}}{\mu}
\end{align}

For Output Resistance,
\begin{align}
    R_{of}=R_{o}(1+GH) \simeq GH R_{o}
\end{align}
\begin{align}
    R_{of} \simeq \mu (\frac{R_{i}}{R_{2}})(g_{m} r_{o 2})(R_{1} \| R_{2})
\end{align}
\begin{align}
    R_{out} = R_{of}=\mu \frac{R_{i}}{R_{1}+R_{2}}(g_{m} r_{o 2}) R_{1}
\end{align}


\item
Given the following values
\begin{table}[!ht]
\centering
%%%%%%%%%%%%%%%%%%%%%%%%%%%%%%%%%%%%%%%%%%%%%%%%%%%%%%%%%%%%%%%%%%%%%%
%%                                                                  %%
%%  This is the header of a LaTeX2e file exported from Gnumeric.    %%
%%                                                                  %%
%%  This file can be compiled as it stands or included in another   %%
%%  LaTeX document. The table is based on the longtable package so  %%
%%  the longtable options (headers, footers...) can be set in the   %%
%%  preamble section below (see PRAMBLE).                           %%
%%                                                                  %%
%%  To include the file in another, the following two lines must be %%
%%  in the including file:                                          %%
%%        \def\inputGnumericTable{}                                 %%
%%  at the beginning of the file and:                               %%
%%        \input{name-of-this-file.tex}                             %%
%%  where the table is to be placed. Note also that the including   %%
%%  file must use the following packages for the table to be        %%
%%  rendered correctly:                                             %%
%%    \usepackage[latin1]{inputenc}                                 %%
%%    \usepackage{color}                                            %%
%%    \usepackage{array}                                            %%
%%    \usepackage{longtable}                                        %%
%%    \usepackage{calc}                                             %%
%%    \usepackage{multirow}                                         %%
%%    \usepackage{hhline}                                           %%
%%    \usepackage{ifthen}                                           %%
%%  optionally (for landscape tables embedded in another document): %%
%%    \usepackage{lscape}                                           %%
%%                                                                  %%
%%%%%%%%%%%%%%%%%%%%%%%%%%%%%%%%%%%%%%%%%%%%%%%%%%%%%%%%%%%%%%%%%%%%%%



%%  This section checks if we are begin input into another file or  %%
%%  the file will be compiled alone. First use a macro taken from   %%
%%  the TeXbook ex 7.7 (suggestion of Han-Wen Nienhuys).            %%
\def\ifundefined#1{\expandafter\ifx\csname#1\endcsname\relax}


%%  Check for the \def token for inputed files. If it is not        %%
%%  defined, the file will be processed as a standalone and the     %%
%%  preamble will be used.                                          %%
\ifundefined{inputGnumericTable}

%%  We must be able to close or not the document at the end.        %%
	\def\gnumericTableEnd{\end{document}}


%%%%%%%%%%%%%%%%%%%%%%%%%%%%%%%%%%%%%%%%%%%%%%%%%%%%%%%%%%%%%%%%%%%%%%
%%                                                                  %%
%%  This is the PREAMBLE. Change these values to get the right      %%
%%  paper size and other niceties.                                  %%
%%                                                                  %%
%%%%%%%%%%%%%%%%%%%%%%%%%%%%%%%%%%%%%%%%%%%%%%%%%%%%%%%%%%%%%%%%%%%%%%

	\documentclass[12pt%
			  %,landscape%
                    ]{report}
       \usepackage[latin1]{inputenc}
       \usepackage{fullpage}
       \usepackage{color}
       \usepackage{array}
       \usepackage{longtable}
       \usepackage{calc}
       \usepackage{multirow}
       \usepackage{hhline}
       \usepackage{ifthen}

	\begin{document}


%%  End of the preamble for the standalone. The next section is for %%
%%  documents which are included into other LaTeX2e files.          %%
\else

%%  We are not a stand alone document. For a regular table, we will %%
%%  have no preamble and only define the closing to mean nothing.   %%
    \def\gnumericTableEnd{}

%%  If we want landscape mode in an embedded document, comment out  %%
%%  the line above and uncomment the two below. The table will      %%
%%  begin on a new page and run in landscape mode.                  %%
%       \def\gnumericTableEnd{\end{landscape}}
%       \begin{landscape}


%%  End of the else clause for this file being \input.              %%
\fi

%%%%%%%%%%%%%%%%%%%%%%%%%%%%%%%%%%%%%%%%%%%%%%%%%%%%%%%%%%%%%%%%%%%%%%
%%                                                                  %%
%%  The rest is the gnumeric table, except for the closing          %%
%%  statement. Changes below will alter the table's appearance.     %%
%%                                                                  %%
%%%%%%%%%%%%%%%%%%%%%%%%%%%%%%%%%%%%%%%%%%%%%%%%%%%%%%%%%%%%%%%%%%%%%%

\providecommand{\gnumericmathit}[1]{#1} 
%%  Uncomment the next line if you would like your numbers to be in %%
%%  italics if they are italizised in the gnumeric table.           %%
%\renewcommand{\gnumericmathit}[1]{\mathit{#1}}
\providecommand{\gnumericPB}[1]%
{\let\gnumericTemp=\\#1\let\\=\gnumericTemp\hspace{0pt}}
 \ifundefined{gnumericTableWidthDefined}
        \newlength{\gnumericTableWidth}
        \newlength{\gnumericTableWidthComplete}
        \newlength{\gnumericMultiRowLength}
        \global\def\gnumericTableWidthDefined{}
 \fi
%% The following setting protects this code from babel shorthands.  %%
 \ifthenelse{\isundefined{\languageshorthands}}{}{\languageshorthands{english}}
%%  The default table format retains the relative column widths of  %%
%%  gnumeric. They can easily be changed to c, r or l. In that case %%
%%  you may want to comment out the next line and uncomment the one %%
%%  thereafter                                                      %%
\providecommand\gnumbox{\makebox[0pt]}
%%\providecommand\gnumbox[1][]{\makebox}

%% to adjust positions in multirow situations                       %%
\setlength{\bigstrutjot}{\jot}
\setlength{\extrarowheight}{\doublerulesep}

%%  The \setlongtables command keeps column widths the same across  %%
%%  pages. Simply comment out next line for varying column widths.  %%
\setlongtables

\setlength\gnumericTableWidth{%
	53pt+%
	93pt+%
0pt}
\def\gumericNumCols{2}
\setlength\gnumericTableWidthComplete{\gnumericTableWidth+%
         \tabcolsep*\gumericNumCols*2+\arrayrulewidth*\gumericNumCols}
\ifthenelse{\lengthtest{\gnumericTableWidthComplete > \linewidth}}%
         {\def\gnumericScale{\ratio{\linewidth-%
                        \tabcolsep*\gumericNumCols*2-%
                        \arrayrulewidth*\gumericNumCols}%
{\gnumericTableWidth}}}%
{\def\gnumericScale{1}}

%%%%%%%%%%%%%%%%%%%%%%%%%%%%%%%%%%%%%%%%%%%%%%%%%%%%%%%%%%%%%%%%%%%%%%
%%                                                                  %%
%% The following are the widths of the various columns. We are      %%
%% defining them here because then they are easier to change.       %%
%% Depending on the cell formats we may use them more than once.    %%
%%                                                                  %%
%%%%%%%%%%%%%%%%%%%%%%%%%%%%%%%%%%%%%%%%%%%%%%%%%%%%%%%%%%%%%%%%%%%%%%

\ifthenelse{\isundefined{\gnumericColA}}{\newlength{\gnumericColA}}{}\settowidth{\gnumericColA}{\begin{tabular}{@{}p{53pt*\gnumericScale}@{}}x\end{tabular}}
\ifthenelse{\isundefined{\gnumericColB}}{\newlength{\gnumericColB}}{}\settowidth{\gnumericColB}{\begin{tabular}{@{}p{93pt*\gnumericScale}@{}}x\end{tabular}}

\begin{tabular}[c]{%
	b{\gnumericColA}%
	b{\gnumericColB}%
	}

%%%%%%%%%%%%%%%%%%%%%%%%%%%%%%%%%%%%%%%%%%%%%%%%%%%%%%%%%%%%%%%%%%%%%%
%%  The longtable options. (Caption, headers... see Goosens, p.124) %%
%	\caption{The Table Caption.}             \\	%
% \hline	% Across the top of the table.
%%  The rest of these options are table rows which are placed on    %%
%%  the first, last or every page. Use \multicolumn if you want.    %%

%%  Header for the first page.                                      %%
%	\multicolumn{2}{c}{The First Header} \\ \hline 
%	\multicolumn{1}{c}{colTag}	%Column 1
%	&\multicolumn{1}{c}{colTag}	\\ \hline %Last column
%	\endfirsthead

%%  The running header definition.                                  %%
%	\hline
%	\multicolumn{2}{l}{\ldots\small\slshape continued} \\ \hline
%	\multicolumn{1}{c}{colTag}	%Column 1
%	&\multicolumn{1}{c}{colTag}	\\ \hline %Last column
%	\endhead

%%  The running footer definition.                                  %%
%	\hline
%	\multicolumn{2}{r}{\small\slshape continued\ldots} \\
%	\endfoot

%%  The ending footer definition.                                   %%
%	\multicolumn{2}{c}{That's all folks} \\ \hline 
%	\endlastfoot
%%%%%%%%%%%%%%%%%%%%%%%%%%%%%%%%%%%%%%%%%%%%%%%%%%%%%%%%%%%%%%%%%%%%%%

\hhline{|-|-}
	 \multicolumn{1}{|p{\gnumericColA}|}%
	{\gnumericPB{\centering}\gnumbox{\textbf{Parameter}}}
	&\multicolumn{1}{p{\gnumericColB}|}%
	{\gnumericPB{\centering}\gnumbox{\textbf{Value}}}
\\
\hhline{|--|}
	 \multicolumn{1}{|p{\gnumericColA}|}%
	{\gnumericPB{\raggedright}\gnumbox[l]{\mu}}
	&\multicolumn{1}{p{\gnumericColB}|}%
	{\gnumericPB{\raggedright}\gnumbox[l]{1000}}
\\
\hhline{|--|}
	 \multicolumn{1}{|p{\gnumericColA}|}%
	{\gnumericPB{\raggedright}\gnumbox[l]{$R_{s}$}}
	&\multicolumn{1}{p{\gnumericColB}|}%
	{\gnumericPB{\raggedright}\gnumbox[l]{\infty}}
\\
\hhline{|--|}
	 \multicolumn{1}{|p{\gnumericColA}|}%
	{\gnumericPB{\raggedright}\gnumbox[l]{$R_{id}$}}
	&\multicolumn{1}{p{\gnumericColB}|}%
	{\gnumericPB{\raggedright}\gnumbox[l]{\infty}}
\\
\hhline{|--|}
	 \multicolumn{1}{|p{\gnumericColA}|}%
	{\gnumericPB{\raggedright}\gnumbox[l]{$r_{o1}$}}
	&\multicolumn{1}{p{\gnumericColB}|}%
	{\gnumericPB{\raggedright}\gnumbox[l]{1k\ohm}}
\\
\hhline{|--|}
	 \multicolumn{1}{|p{\gnumericColA}|}%
	{\gnumericPB{\raggedright}\gnumbox[l]{$R_{1}$}}
	&\multicolumn{1}{p{\gnumericColB}|}%
	{\gnumericPB{\raggedright}\gnumbox[l]{10k\ohm}}
\\
\hhline{|--|}
	 \multicolumn{1}{|p{\gnumericColA}|}%
	{\gnumericPB{\raggedright}\gnumbox[l]{$R_{2}$}}
	&\multicolumn{1}{p{\gnumericColB}|}%
	{\gnumericPB{\raggedright}\gnumbox[l]{90k\ohm}}
\\
\hhline{|--|}
	 \multicolumn{1}{|p{\gnumericColA}|}%
	{\gnumericPB{\raggedright}\gnumbox[l]{g_{m}}}
	&\multicolumn{1}{p{\gnumericColB}|}%
	{\gnumericPB{\raggedright}\gnumbox[l]{5mA/V}}
\\
\hhline{|--|}
	 \multicolumn{1}{|p{\gnumericColA}|}%
	{\gnumericPB{\raggedright}\gnumbox[l]{r_{o2}}}
	&\multicolumn{1}{p{\gnumericColB}|}%
	{\gnumericPB{\raggedright}\gnumbox[l]{20k\ohm}}
\\
\hhline{|-|-|}
\end{tabular}

\ifthenelse{\isundefined{\languageshorthands}}{}{\languageshorthands{\languagename}}
\gnumericTableEnd
\caption{}
\label{table: Input_Table}
\end{table}

Find numerical value of $R_{i}$ and use it to find the value of G

\solution
Using the given numerical values on the previously obtained equations, we obtain:
\begin{align}
    R_{i}=\infty\|\infty\|(10+90)=100 k\ohm
\end{align}

\begin{align}
    G =-1000 \frac{100}{10 \| 90}=-11.11 \times 10^{3}
\end{align}

\item 
Check the validity of the approximation that we use to neglect $1/g_{m}$

\solution
\begin{align}
    1 / g_{m}=0.2 k\ohm \ll (10\|90\| 20)k\ohm = 6.2k\ohm
\end{align}
Hence, we can see that our approximation is valid

\item
Find the value of feedback gain H and open loop gain GH

\solution
\begin{align}
    H=-\frac{R_{1}}{R_{1}+R_{2}}=-\frac{10}{10+90}=-0.1
\end{align}

\begin{align}
    GH=1111 \gg 1
\end{align}

\item
Find the approximate value of closed loop gain T

\solution
\begin{align}
    T \simeq \frac{1}{H} = -\frac{1}{0.1} = -10
\end{align}

\item
Find the values of $R_{in}$ and $R_{out}$

\solution
\begin{align}
    R_{in}=\frac{R_{2}}{\mu}=\frac{90k\ohm}{1000}=90\ohm
\end{align}
\begin{align}
    R_{o} &=g_{m} r_{o 2}(R_{1} \| R_{2}) =5 \times 20(10 \| 90)=900k\ohm
\end{align}
\begin{align}
    R_{out}=(1+GH) R_{o}=1112 \times 900 \simeq 1000M\ohm
\end{align}

\begin{table}[!ht]
\centering
%%%%%%%%%%%%%%%%%%%%%%%%%%%%%%%%%%%%%%%%%%%%%%%%%%%%%%%%%%%%%%%%%%%%%%
%%                                                                  %%
%%  This is the header of a LaTeX2e file exported from Gnumeric.    %%
%%                                                                  %%
%%  This file can be compiled as it stands or included in another   %%
%%  LaTeX document. The table is based on the longtable package so  %%
%%  the longtable options (headers, footers...) can be set in the   %%
%%  preamble section below (see PRAMBLE).                           %%
%%                                                                  %%
%%  To include the file in another, the following two lines must be %%
%%  in the including file:                                          %%
%%        \def\inputGnumericTable{}                                 %%
%%  at the beginning of the file and:                               %%
%%        \input{name-of-this-file.tex}                             %%
%%  where the table is to be placed. Note also that the including   %%
%%  file must use the following packages for the table to be        %%
%%  rendered correctly:                                             %%
%%    \usepackage[latin1]{inputenc}                                 %%
%%    \usepackage{color}                                            %%
%%    \usepackage{array}                                            %%
%%    \usepackage{longtable}                                        %%
%%    \usepackage{calc}                                             %%
%%    \usepackage{multirow}                                         %%
%%    \usepackage{hhline}                                           %%
%%    \usepackage{ifthen}                                           %%
%%  optionally (for landscape tables embedded in another document): %%
%%    \usepackage{lscape}                                           %%
%%                                                                  %%
%%%%%%%%%%%%%%%%%%%%%%%%%%%%%%%%%%%%%%%%%%%%%%%%%%%%%%%%%%%%%%%%%%%%%%



%%  This section checks if we are begin input into another file or  %%
%%  the file will be compiled alone. First use a macro taken from   %%
%%  the TeXbook ex 7.7 (suggestion of Han-Wen Nienhuys).            %%
\def\ifundefined#1{\expandafter\ifx\csname#1\endcsname\relax}


%%  Check for the \def token for inputed files. If it is not        %%
%%  defined, the file will be processed as a standalone and the     %%
%%  preamble will be used.                                          %%
\ifundefined{inputGnumericTable}

%%  We must be able to close or not the document at the end.        %%
	\def\gnumericTableEnd{\end{document}}


%%%%%%%%%%%%%%%%%%%%%%%%%%%%%%%%%%%%%%%%%%%%%%%%%%%%%%%%%%%%%%%%%%%%%%
%%                                                                  %%
%%  This is the PREAMBLE. Change these values to get the right      %%
%%  paper size and other niceties.                                  %%
%%                                                                  %%
%%%%%%%%%%%%%%%%%%%%%%%%%%%%%%%%%%%%%%%%%%%%%%%%%%%%%%%%%%%%%%%%%%%%%%

	\documentclass[12pt%
			  %,landscape%
                    ]{report}
       \usepackage[latin1]{inputenc}
       \usepackage{fullpage}
       \usepackage{color}
       \usepackage{array}
       \usepackage{longtable}
       \usepackage{calc}
       \usepackage{multirow}
       \usepackage{hhline}
       \usepackage{ifthen}

	\begin{document}


%%  End of the preamble for the standalone. The next section is for %%
%%  documents which are included into other LaTeX2e files.          %%
\else

%%  We are not a stand alone document. For a regular table, we will %%
%%  have no preamble and only define the closing to mean nothing.   %%
    \def\gnumericTableEnd{}

%%  If we want landscape mode in an embedded document, comment out  %%
%%  the line above and uncomment the two below. The table will      %%
%%  begin on a new page and run in landscape mode.                  %%
%       \def\gnumericTableEnd{\end{landscape}}
%       \begin{landscape}


%%  End of the else clause for this file being \input.              %%
\fi

%%%%%%%%%%%%%%%%%%%%%%%%%%%%%%%%%%%%%%%%%%%%%%%%%%%%%%%%%%%%%%%%%%%%%%
%%                                                                  %%
%%  The rest is the gnumeric table, except for the closing          %%
%%  statement. Changes below will alter the table's appearance.     %%
%%                                                                  %%
%%%%%%%%%%%%%%%%%%%%%%%%%%%%%%%%%%%%%%%%%%%%%%%%%%%%%%%%%%%%%%%%%%%%%%

\providecommand{\gnumericmathit}[1]{#1} 
%%  Uncomment the next line if you would like your numbers to be in %%
%%  italics if they are italizised in the gnumeric table.           %%
%\renewcommand{\gnumericmathit}[1]{\mathit{#1}}
\providecommand{\gnumericPB}[1]%
{\let\gnumericTemp=\\#1\let\\=\gnumericTemp\hspace{0pt}}
 \ifundefined{gnumericTableWidthDefined}
        \newlength{\gnumericTableWidth}
        \newlength{\gnumericTableWidthComplete}
        \newlength{\gnumericMultiRowLength}
        \global\def\gnumericTableWidthDefined{}
 \fi
%% The following setting protects this code from babel shorthands.  %%
 \ifthenelse{\isundefined{\languageshorthands}}{}{\languageshorthands{english}}
%%  The default table format retains the relative column widths of  %%
%%  gnumeric. They can easily be changed to c, r or l. In that case %%
%%  you may want to comment out the next line and uncomment the one %%
%%  thereafter                                                      %%
\providecommand\gnumbox{\makebox[0pt]}
%%\providecommand\gnumbox[1][]{\makebox}

%% to adjust positions in multirow situations                       %%
\setlength{\bigstrutjot}{\jot}
\setlength{\extrarowheight}{\doublerulesep}

%%  The \setlongtables command keeps column widths the same across  %%
%%  pages. Simply comment out next line for varying column widths.  %%
\setlongtables

\setlength\gnumericTableWidth{%
	53pt+%
	93pt+%
0pt}
\def\gumericNumCols{2}
\setlength\gnumericTableWidthComplete{\gnumericTableWidth+%
         \tabcolsep*\gumericNumCols*2+\arrayrulewidth*\gumericNumCols}
\ifthenelse{\lengthtest{\gnumericTableWidthComplete > \linewidth}}%
         {\def\gnumericScale{\ratio{\linewidth-%
                        \tabcolsep*\gumericNumCols*2-%
                        \arrayrulewidth*\gumericNumCols}%
{\gnumericTableWidth}}}%
{\def\gnumericScale{1}}

%%%%%%%%%%%%%%%%%%%%%%%%%%%%%%%%%%%%%%%%%%%%%%%%%%%%%%%%%%%%%%%%%%%%%%
%%                                                                  %%
%% The following are the widths of the various columns. We are      %%
%% defining them here because then they are easier to change.       %%
%% Depending on the cell formats we may use them more than once.    %%
%%                                                                  %%
%%%%%%%%%%%%%%%%%%%%%%%%%%%%%%%%%%%%%%%%%%%%%%%%%%%%%%%%%%%%%%%%%%%%%%

\ifthenelse{\isundefined{\gnumericColA}}{\newlength{\gnumericColA}}{}\settowidth{\gnumericColA}{\begin{tabular}{@{}p{53pt*\gnumericScale}@{}}x\end{tabular}}
\ifthenelse{\isundefined{\gnumericColB}}{\newlength{\gnumericColB}}{}\settowidth{\gnumericColB}{\begin{tabular}{@{}p{93pt*\gnumericScale}@{}}x\end{tabular}}

\begin{tabular}[c]{%
	b{\gnumericColA}%
	b{\gnumericColB}%
	}

%%%%%%%%%%%%%%%%%%%%%%%%%%%%%%%%%%%%%%%%%%%%%%%%%%%%%%%%%%%%%%%%%%%%%%
%%  The longtable options. (Caption, headers... see Goosens, p.124) %%
%	\caption{The Table Caption.}             \\	%
% \hline	% Across the top of the table.
%%  The rest of these options are table rows which are placed on    %%
%%  the first, last or every page. Use \multicolumn if you want.    %%

%%  Header for the first page.                                      %%
%	\multicolumn{2}{c}{The First Header} \\ \hline 
%	\multicolumn{1}{c}{colTag}	%Column 1
%	&\multicolumn{1}{c}{colTag}	\\ \hline %Last column
%	\endfirsthead

%%  The running header definition.                                  %%
%	\hline
%	\multicolumn{2}{l}{\ldots\small\slshape continued} \\ \hline
%	\multicolumn{1}{c}{colTag}	%Column 1
%	&\multicolumn{1}{c}{colTag}	\\ \hline %Last column
%	\endhead

%%  The running footer definition.                                  %%
%	\hline
%	\multicolumn{2}{r}{\small\slshape continued\ldots} \\
%	\endfoot

%%  The ending footer definition.                                   %%
%	\multicolumn{2}{c}{That's all folks} \\ \hline 
%	\endlastfoot
%%%%%%%%%%%%%%%%%%%%%%%%%%%%%%%%%%%%%%%%%%%%%%%%%%%%%%%%%%%%%%%%%%%%%%

\hhline{|-|-}
	 \multicolumn{1}{|p{\gnumericColA}|}%
	{\gnumericPB{\centering}\gnumbox{\textbf{Parameter}}}
	&\multicolumn{1}{p{\gnumericColB}|}%
	{\gnumericPB{\centering}\gnumbox{\textbf{Value}}}
\\
\hhline{|--|}
	 \multicolumn{1}{|p{\gnumericColA}|}%
	{\gnumericPB{\raggedright}\gnumbox[l]{$R_{i}$}}
	&\multicolumn{1}{p{\gnumericColB}|}%
	{\gnumericPB{\raggedright}\gnumbox[l]{$100k\Omega$}}
\\
\hhline{|--|}
	 \multicolumn{1}{|p{\gnumericColA}|}%
	{\gnumericPB{\raggedright}\gnumbox[l]{$1/g_{m}$}}
	&\multicolumn{1}{p{\gnumericColB}|}%
	{\gnumericPB{\raggedright}\gnumbox[l]{$200\Omega$}}
\\
\hhline{|--|}
	 \multicolumn{1}{|p{\gnumericColA}|}%
	{\gnumericPB{\raggedright}\gnumbox[l]{$G$}}
	&\multicolumn{1}{p{\gnumericColB}|}%
	{\gnumericPB{\raggedright}\gnumbox[l]{$-1.11 \times 10^4$}}
\\
\hhline{|--|}
	 \multicolumn{1}{|p{\gnumericColA}|}%
	{\gnumericPB{\raggedright}\gnumbox[l]{$H$}}
	&\multicolumn{1}{p{\gnumericColB}|}%
	{\gnumericPB{\raggedright}\gnumbox[l]{$-0.1$}}
\\
\hhline{|--|}
	 \multicolumn{1}{|p{\gnumericColA}|}%
	{\gnumericPB{\raggedright}\gnumbox[l]{$GH$}}
	&\multicolumn{1}{p{\gnumericColB}|}%
	{\gnumericPB{\raggedright}\gnumbox[l]{$1111$}}
\\
\hhline{|--|}
	 \multicolumn{1}{|p{\gnumericColA}|}%
	{\gnumericPB{\raggedright}\gnumbox[l]{$T$}}
	&\multicolumn{1}{p{\gnumericColB}|}%
	{\gnumericPB{\raggedright}\gnumbox[l]{$-10$}}
\\
\hhline{|--|}
	 \multicolumn{1}{|p{\gnumericColA}|}%
	{\gnumericPB{\raggedright}\gnumbox[l]{$R_{in}$}}
	&\multicolumn{1}{p{\gnumericColB}|}%
	{\gnumericPB{\raggedright}\gnumbox[l]{$90\Omega$}}
\\
\hhline{|--|}
	 \multicolumn{1}{|p{\gnumericColA}|}%
	{\gnumericPB{\raggedright}\gnumbox[l]{$R_{o}$}}
	&\multicolumn{1}{p{\gnumericColB}|}%
	{\gnumericPB{\raggedright}\gnumbox[l]{$900k\Omega$}}
\\
\hhline{|--|}
	 \multicolumn{1}{|p{\gnumericColA}|}%
	{\gnumericPB{\raggedright}\gnumbox[l]{$R_{out}$}}
	&\multicolumn{1}{p{\gnumericColB}|}%
	{\gnumericPB{\raggedright}\gnumbox[l]{$1000M\Omega$}}
\\
\hhline{|-|-|}
\end{tabular}

\ifthenelse{\isundefined{\languageshorthands}}{}{\languageshorthands{\languagename}}
\gnumericTableEnd

\caption{}
\label{table: Output_Table}
\end{table}

\item
Verify the above calculations using a Python code.

\solution
\begin{lstlisting}
codes/ee18btech11021/ee18btech11021_calc.py
\end{lstlisting}

\end{enumerate}
